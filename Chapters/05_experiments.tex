\documentclass[../main.tex]{subfiles}
\graphicspath{{\subfix{../Figures/}}}

\begin{document}

\chapter{Experiments}

\section{Models}

All models are trained using the Adam algorithm \cite{kingmaAdam2014} with a learning rate of $1 \times 10^{-3}$.

\subsection{Classifier}

The classifier architecture in our analyses is an MLP consisting of two 50-node linear layers each followed by a $\relu$, followed by one last linear layer outputting $\outputdim$ logits.

\subsection{Autoencoder}

We build a normalizing flow model according to the NICE architecture \cite{dinhNICE2015}.
The model is trained according to a Gaussian prior distribution, and its layers are 4 additive coupling layers where we divide the input into the odd columns and the even columns, and the nonlinearity is an MLP with 4 layers of 50 nodes each \note{check this is the case}.

\section{Validity losses}

In this experiment, the goal is to find the validity loss that maximizes validity rate among a number of choices.

\subsection{Description}
\label{validity_losses/description}

For a given dataset, we train one classifier.
Then for this dataset, we train several autoencoders all with the same architecture and with different seeds.
The seeds are generated uniformly based on one primary seed.

For a given dataset, we measure the validity rate across the whole test set.
That is, for every test point, we run the classifier to get the $\source$ class, and we sample the target class uniformly from $\setclasses \backslash \{\source\}$.
Then we generate a path using the given validity loss, and measure its validity.

We average the rates over the different seeds, and report the standard error.

The parameters used for training the models and for producing the paths are reported in ?? \note{Link to some yaml?}

\subsection{Baselines}

The paths are produced by variations on \ls{} and with \revise{}.
Specifically, we run both methods twice: once with regularization on the $L_1$ distance, as explained in \autoref{par:revise}, and one without regularization (with the $\lambda_\text{distance}$ set to 0).
\note{How are these named in the results table?}

The losses compared in this experiment are as follows:
\note{todo}

\subsection{Datasets}

We run our experiment on the datasets described in \autoref{sec:datasets}, namely \CakeOnSea, \ForestCover, \WineQuality{} and \OnlineNewsPopularity.

\subsection{Metrics}

The one metric in this experiment is the validity rate, \ie{} the proportion of paths that reach their target class across the whole test set.

\subsection{Results}

The results for each dataset and loss are given in \autoref{tbl:validity_losses}.

\begin{table}[h!]
    \caption{Validity rate means with their standard error.}
\label{tbl:validity_losses}
    \begin{tabular}{lllll}
data\_module & \multicolumn{2}{r}{CakeOnSeaDataset} & \multicolumn{2}{r}{ForestCoverDataset} \\
path\_method & LatentShift & Revise & LatentShift & Revise \\
loss &  &  &  &  \\
BinaryStretchLoss & $0.5276±0.0000$ & 0.5276±0.0000 & 0.3137±0.0070 & 0.2791±0.0010 \\
StretchLoss & 0.5276±0.0000 & 0.5276±0.0000 & 0.2994±0.0221 & 0.2815±0.0019 \\
TargetLoss & 0.5276±0.0000 & 0.5276±0.0000 & 0.3095±0.0025 & 0.2833±0.0001 \\
\end{tabular}

\begin{tabular}{lllllll}
     & \multicolumn{2}{r}{CakeOnSeaDataset} & \multicolumn{2}{r}{ForestCoverDataset} & \multicolumn{2}{r}{WineQualityDataset} \\
    & LatentShift & Revise & LatentShift & Revise & LatentShift & Revise \\
    BinaryStretchLoss & 44.9±1.1 & 57.6±0.8 & 51.9±4.2 & 97.6±0.3 & 91.0±0.5 & 99.5±0.2 \\
    StretchLoss & 29.7±1.9 & 51.7±0.7 & 26.2±1.7 & 30.8±2.6 & 73.9±0.8 & 77.9±0.1 \\
    TargetLoss & 45.2±1.2 & 57.8±0.7 & 52.6±4.2 & 97.5±0.2 & 91.7±0.5 & 99.4±0.2 \\
\end{tabular}
    


\begin{tabular}{lllllllllllll}
    Dataset & \multicolumn{4}{r}{CakeOnSea} & \multicolumn{4}{r}{ForestCover} & \multicolumn{4}{r}{WineQuality} \\
    Path method & LatentShift & LatentShiftNoDistanceReg & Revise & ReviseNoDistanceReg & LatentShift & LatentShiftNoDistanceReg & Revise & ReviseNoDistanceReg & LatentShift & LatentShiftNoDistanceReg & Revise & ReviseNoDistanceReg \\
    Loss function &  &  &  &  &  &  &  &  &  &  &  &  \\
    TargetLoss & 45.2±1.2 & 45.0±1.3 & 71.2±2.7 & 71.3±2.5 & 53.1±4.1 & 52.9±4.1 & 99.8±0.0 & 99.8±0.0 & 92.4±0.6 & 92.4±0.8 & 100.0±0.0 & 100.0±0.0 \\
    BinaryStretchLoss & 46.2±0.9 & 46.2±0.9 & 90.1±1.3 & 90.2±1.2 & 55.9±4.3 & 55.8±4.0 & 99.6±0.1 & 99.6±0.1 & 91.8±0.7 & 91.8±0.3 & 100.0±0.0 & 99.9±0.1 \\
    StretchLoss & 34.1±2.3 & 34.2±2.2 & 73.5±8.1 & 73.6±7.8 & 33.2±1.8 & 33.1±1.9 & 30.6±1.0 & 30.4±1.0 & 75.4±0.5 & 75.8±0.6 & 78.1±0.3 & 77.9±0.3 \\
\end{tabular}
    
\end{table}

\section{Path regularized training}

Having determined an appropriate validity loss to use for generating valid paths, we turn now to our objective of interpretability.
In this experiment we seek to determine if \ls{} equipped with a path regularized autoencoder can perform on the same level as \revise{}.




The parameters used for training the models and for producing the paths are reported in ?? \note{Link to some yaml?}

\subsection{Description}

First, we generate some seeds.
For a given dataset, we train one classifier, as in \autoref{validity_losses/description}.
For each seed, we train several autoencoders: one without path regularization, and one per CF generation method.

% {
%     \newcommand{\var}[1]{\mathrm{#1}}
%     \newcommand{\fn}[1]{\mathtt{#1}}

%     \newcommand{\seed}{\var{seed}}
%     \newcommand{\seeds}{\var{seeds}}
%     \newcommand{\nbSeeds}{N_\seeds}

%     \newcommand{\varDataset}{\var{dataset}}
%     \newcommand{\varDatasets}{\var{datasets}}

%     \newcommand{\explainer}{\var{explainer}}
%     \newcommand{\explainers}{\var{explainers}}

%     \newcommand{\autoencoders}{\var{autoencoders}}

%     \newcommand{\metrics}{\var{metrics}}
%     \newcommand{\aggregatedMetrics}{\var{aggregated\_metrics}}

%     \newcommand{\results}{\var{results}}

%     \newcommand{\generateSeeds}{\fn{generate\_seeds}}
%     \newcommand{\trainModel}{\fn{train\_model}}
%     \newcommand{\newAutoencoder}{\fn{new\_autoencoder}}
%     \newcommand{\randomTarget}{\fn{random\_target}}
%     \newcommand{\generatePath}{\fn{generate\_path}}
%     \newcommand{\computeMetrics}{\fn{compute\_metrics}}
%     \newcommand{\aggregateMetrics}{\fn{aggregate\_metrics}}

%     \begin{algorithm}
%         \caption{Experiment on path regularization}
%         \label{algo:pathreg_xp}
%         \KwData{$\seed$, number of seeds $\nbSeeds$, $\varDatasets$, $\explainers$}
%         \KwOut{}
%         $\seeds \gets \generateSeeds(\nbSeeds, \seed)$ \\

%         \ForEach{$\varDataset \in \varDatasets$}{
%             $\trainModel(\varDataset, f)$ \\
%             $\autoencoders = \emptyset$ \\

%             $\results \gets \emptyset$ \\

%             \ForEach{$\seed \in \seeds$}{
%                 $\autoencoder \gets \newAutoencoder(\seed, null)$
%                 \tcp*{no path regularization}
%                 $\trainModel(\varDataset, \autoencoder)$ \\
%                 $\autoencoders = \autoencoders \cup \{\autoencoder\}$ \\

%                 \ForEach{$\explainer \in \explainers$}{
%                     $\autoencoder \gets \newAutoencoder(\seed, \explainer)$ \\
%                     $\trainModel_\explainer(\varDataset, \autoencoder)$ \\

%                     $\autoencoders = \autoencoders \cup \{ \autoencoder \}$ \\
%                 }
%             }

%             \seedMetrics \gets \emptyset \\
%             \ForEach{$\autoencoder \in \autoencoders$}{
%                 $\metrics = \emptyset$ \\
%                 \ForEach{$x \in \testset$}{
%                     $\target \gets \randomTarget(x)$ \\
%                     $\apath \gets \generatePath(x, \target)$ \\
%                     $\metrics \gets \metrics \cup \computeMetrics(\apath, \target)$ \\
%                 }
%                 $\aggregatedMetrics = \aggregateMetrics(\metrics)$  \\
%             }

%         }
%     \end{algorithm}
% }

\subsection{Baselines}

\subsection{Datasets}

\subsection{Metrics}


\subsection{Results}

\end{document}