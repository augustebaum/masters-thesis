\documentclass[../main.tex]{subfiles}
\graphicspath{{\subfix{../Figures/}}}

\begin{document}

\chapter{Experiments}

\section{Models}

All models are trained using the Adam algorithm \cite{kingmaAdam2014} with a learning rate of $1 \times 10^{-3}$.

\subsection{Classifier}

The classifier architecture in our analyses is an MLP consisting of two 50-node linear layers each followed by a $\relu$, followed by one last linear layer outputting $\outputdim$ logits.

\subsection{Autoencoder}

We build a normalizing flow model according to the NICE architecture \cite{dinhNICE2015}.
The model is trained according to a Gaussian prior distribution, and its layers are 4 additive coupling layers where we divide the input into the odd columns and the even columns, and the nonlinearity is an MLP with 4 layers of 50 nodes each.

\section{Validity losses}

The goal is to find the best validity loss, the one that maximizes validity rate.

\subsection{Experimental setup}

For a given dataset, we train one classifier.
Then for this dataset, we train several autoencoders all with the same architecture and with different seeds.
The seeds are generated uniformly based on one primary seed.

For a given dataset, we measure the validity rate across the whole test set.
That is, for every test point, we run the classifier to get the $\source$ class, and we sample the target class uniformly from $\setclasses \backslash \{\source\}$.
Then we generate a path using the given validity loss, and measure its validity.
We do this with \ls{} and with \revise{}.

We average the rates over the different autoencoders, and report the standard error.

The parameters used are reported in ?? \citenote{}

\subsection{Results}

\begin{table}
    \caption{Validity rate means with their standard error.}
    \begin{tabular}{lllll}
data\_module & \multicolumn{2}{r}{CakeOnSeaDataset} & \multicolumn{2}{r}{ForestCoverDataset} \\
path\_method & LatentShift & Revise & LatentShift & Revise \\
loss &  &  &  &  \\
BinaryStretchLoss & $0.5276±0.0000$ & 0.5276±0.0000 & 0.3137±0.0070 & 0.2791±0.0010 \\
StretchLoss & 0.5276±0.0000 & 0.5276±0.0000 & 0.2994±0.0221 & 0.2815±0.0019 \\
TargetLoss & 0.5276±0.0000 & 0.5276±0.0000 & 0.3095±0.0025 & 0.2833±0.0001 \\
\end{tabular}

\end{table}

\section{Path regularized training}

\subsection{Experimental setup}

\subsection{Results}

\end{document}