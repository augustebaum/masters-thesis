\documentclass[../main.tex]{subfiles}
\graphicspath{{\subfix{../Figures/}}}

\begin{document}

\chapter{Previous work}
\label{ch:previous_work}

In this chapter we detail some of the topics and works that are either directly relevant to our work, or were otherwise of interest during this project.

% A comprehensive overview of algorithms, evaluation metrics as well as open problems is given in \cite{vermaCounterfactual2020}.

% \section{}

Counterfactual statements are closely related to causal inference \citenote.
This link is formally made in Pearl's theory of \emph{structural causal models} (or SCMs for short): given a complete formal description of the causal relationships between variables, the theory describes a natural procedure to compute how a change in one variable would influence the others. \note{Is this true only for additive noise models? or all SCMS}

Karimi \textsl{et al.~} develop this relationship in the context of CFX in the form of an lower bound statement:
given some cost function, if the true SCM is known fully, then the Pearl procedure yields a CF that has minimum cost.
Any other proposed CF will in some way neglect some of the variable relationships, and hence include redundant perturbations which increase the cost.

For instance, 

\end{document}